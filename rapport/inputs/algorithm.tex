Dans le cadre de ce projet, différentes techniques heuristiques ont été utilisées pour améliorer l'algorithme de prise de décision. 
Nous avons pris la décision de nous concentrer exclusivement sur les heuristiques basées sur l'occupation de l'espace. Ces heuristiques sont particulièrement 
adaptées pour faciliter l'analyse des positions de jeu et l'évaluation de leur qualité. 

\subsection{OpenSpace}
L'heuristique \textbf{OpenSpace} est une technique simple basée sur la comptabilisation du nombre de cases vides entourant 
un joueur donné. Elle se fonde uniquement sur la position de ce joueur et ne prend pas en considération la position des 
autres joueurs.

\tocless\subsubsection{Implémentation}
Son implémentation consiste à calculer le nombre de cases vides situées autour d'un joueur donné, à partir d'un état de jeu spécifique.\\

\begin{algorithm}[H]
	\caption{OpenSpace}
	\label{alg:openspace}
	\DontPrintSemicolon
	\KwIn{$state$ : état du jeu}
	\KwOut{$scores$ : score de l'état}
	$scores \gets \{\}$\;
	\For{$player \in state.players$}{
		$score \gets 0$\;
		\For{$neighbor \in state.neighbors(player)$}{
			\If{$neighbor \in state.empty$}{
				$score \gets score + 1$\;
			}
		}
		$scores[player] \gets score$\;
	}
	\Return $scores$\;
\end{algorithm}

\tocless\subsubsection{Complexité}
L'algorithme \textbf{OpenSpace}~\ref{alg:openspace} a une complexité temporelle qui dépend uniquement du nombre de joueurs dans le jeu.\\

En effet la boucle extérieure de l'algorithme parcourt tous les joueurs présents dans l'état de jeu, ce qui prend un temps 
proportionnel à N, où N est le nombre de joueurs. De plus la boucle intérieure itère sur les quatres(4) voisins de chaque joueur, et 
la vérification de la liberté de chaque voisin(voir la ligne 5~\ref{alg:openspace}) se fait en temps constant($\mathcal{O}(1)$).\\

Par conséquent, la complexité de l'algorithme \textbf{Openspace} est de l'ordre de $\mathcal{O}(N)$, 
ce qui peut être considéré comme linéaire en fonction du nombre de joueurs.\\

\subsection{GASLAP}
\textbf{GASLAP} abréviation de "\textbf{G}o \textbf{AS} \textbf{L}ong \textbf{A}s \textbf{P}ossible" est une approche 
qui consiste à calculer la distance maximale qu'un joueur peut parcourir à partir de sa position de départ dans une 
direction donnée, sans rencontrer d'obstacles. Cette méthode permet de prioriser les mouvements qui permettent au 
joueur de se déplacer le plus loin possible dans la direction choisie, tout en évitant les déplacements inutiles 
et en maximisant l'efficacité des mouvements.

\tocless\subsubsection{Implémentation}
Cette approche s'appuie sur la détermination du nombre de cases vides situées dans la direction choisie par le joueur.\\

\begin{algorithm}[H]
	\caption{GASLAP}
	\label{alg:gaslap}
	\DontPrintSemicolon
	\KwIn{$state$ : état du jeu}
	\KwOut{$scores$ : score de l'état}
	$scores \gets \{\}$\;
	\For{$player \in state.players$}{
		$score \gets 0$\;
		$direction \gets state.direction(player)$\;
		$position \gets player.position$\;
		\While{$position \in state.empty$}{
			$position \gets position + direction$\;
			$score \gets score + 1$\;
		}
		$scores[player] \gets score$\;
	}
	\Return $scores$\;
\end{algorithm}

\tocless\subsubsection{Complexité}
Étant donné que le jeu se déroule sur une grille carrée de taille fixe $n \times n$,un joueur ne peut se déplacer dans une même direction que
$n-1$ fois. Ainsi, la complexité de la boucle intérieure(boucle \textbf{While}) est de l'ordre de $\mathcal{O}(n)$. De plus la boucle extérieure
(boucle \textbf{For}) parcourt tous les joueurs présents dans l'état de jeu, ce qui prend un temps proportionnel à $J$, où $J$ est le nombre de joueurs.\\

Par conséquent, la complexité de l'algorithme \textbf{GASLAP} est de l'ordre de $\mathcal{O}(J \times n)$, ce qui peut être considéré comme
linéaire en fonction du nombre de joueurs car $n$ est une constante.\\

\subsection{Voronoï}
Le diagramme de \textbf{Voronoï} est une technique mathématique qui permet de diviser l'espace en régions distinctes en 
fonction de la proximité de points spécifiques. Dans le cadre de ce projet, nous avons utilisé cette technique pour 
déterminer les zones d'influence\footnote{Zone d'influence est la zone pour laquelle un joueur peut atteindre en premier à partir de sa position de départ.} 
des joueurs.
\tocless\subsubsection{Implémentation}
Pour mettre en œuvre cette technique, nous commençons par calculer la distance entre le joueur et toutes les autres 
cases vides\footnote{Case vide est une case qui n'est pas occupée par un joueur.} de la grille. Pour ce faire, nous adaptons 
les principes de l'algorithme de \textbf{Dijkstra} à notre cas spécifique(voir algorithme \ref{algo:distance}).\\

\begin{algorithm}[H]
	\caption{Détermination de la distance entre le joueur et les cases vides}
	\label{algo:distance}
	\DontPrintSemicolon
	\SetKwComment{Comment}{/* }{ */}
	\KwIn{$initial\_position$ : Position initial du joueur, $positions\_vides$ : Ensemble des positions des cases vides}
	\KwOut{$distances$ : Dictionnaire contenant les distances entre le joueur et les cases vides}
	$distances \gets \{\}$\;
	$queue \gets \{\}$\;
	\ForEach{$position\_vide \in positions\_vides$}{
		$distances[position\_vide] \gets \infty$\;
	}
	$distances[initial\_position] \gets 0$\;
	$queue \gets queue \cup \{initial\_position\}$\;
	\While{$queue \neq \{\}$}{
		$position \gets queue[0]$ \;
		$queue \gets queue \setminus \{position\}$\;
		\ForEach{$adjacent \in get\_adjacents(position)$}{
			\If{$distances[adjacent] > distances[position] + 1$}{
				$distances[adjacent] \gets distances[position] + 1$\;
				$queue \gets queue \cup \{adjacent\}$\;
			}
		}
	}
	\Return $distances$\;
\end{algorithm}

Après avoir calculé les distances entre chaque joueur et les cases vides à l'aide de l'algorithm \ref{algo:distance}, 
nous utilisons ces informations pour déterminer les zones d'influence de chaque joueur en ajoutant les cases vides les plus 
proches de lui, et nous obtenons la taille de ces zones d'influence en utilisant la méthode décrite dans l'algorithme \ref{algo:voronoï}.\\

\begin{algorithm}[H]
	\caption{Détermination de la zone d'influence d'un joueur}
	\label{algo:voronoï}
	\DontPrintSemicolon
	\SetKwComment{Comment}{/* }{ */}
	\KwIn{$state$ : Etat courant du jeu}
	\KwOut{$zones\_influence$ : Dictionnaire contenant la taille de la zone d'influence de chaque joueur}
	$zones\_influence \gets \{\}$\;
	$distances \gets \{\}$\;
	\ForEach{$player \in state.players$}{
		$zones\_influence[player] \gets 0$\;
		$distances[player] \gets distance(player.position, state.vide\_positions)$\;
	}
	\ForEach{$position\_vide \in state.vide\_positions$}{
		$min\_distance \gets \infty$\;
		$min\_player \gets None$\;
		\ForEach{$player \in state.players$}{
			\If{$distances[player][position\_vide] < min\_distance$}{
				$min\_distance \gets distances[player][position\_vide]$ \;
				$min\_player \gets player$\;
			}
			\ElseIf{$distances[player][position\_vide] == min\_distance$}{
				$min\_player \gets None$\;
			}
		}
		\If{$min\_player \neq None$}{
			$zones\_influence[min\_player] \gets zones\_influence[min\_player] + 1$\;
		}
	}
	\Return $zones\_influence$\;
\end{algorithm}

\tocless\subsubsection{Complexité}
Étant donné que l'algorithm \textbf{Voronoï}~\ref{algo:voronoï} est basé sur l'utilisation de l'algorithme de \textbf{distance}~\ref{algo:distance}, 
il existe un lien direct entre la complexité de ces deux algorithmes. \\

En effet, l'algorithme de \textbf{distance}~\ref{algo:distance} est basé sur l'algorithme de \textbf{Dijkstra} qui est un algorithme ayant une complexité
de l'ordre de $\mathcal{O}{(|A| + |S|log|S|)}$ où $A$ est l'ensemble des arêtes \footnote{Arête est une liaison entre deux cases.} de la grille et 
$S$ est l'ensemble des cases vide de la grille. Vu que l'on itère $J$ fois sur l'algorithme de \textbf{distance}~\ref{algo:distance} où $J$ est le nombre de joueurs,
la complexité de la première partie de l'algorithme \textbf{Voronoï}~\ref{algo:voronoï}~(ligne 3-6) est de l'ordre de $\mathcal{O}{(J(|A| + |S|log|S|))}$. La deuxième partie de l'algorithme
\textbf{Voronoï}~\ref{algo:voronoï}~(ligne 7-15) est basée sur une double itération entre les cases vides et les joueurs, la complexité de cette partie est donc de l'ordre de
$\mathcal{O}{(|S|J)}$. \\

La complexité totale de l'algorithme \textbf{Voronoï}~\ref{algo:voronoï} est donc de l'ordre de $\mathcal{O}{(J(|A| + |S|log|S|) + |S|J)}$.\\

Toutefois, en fonction de l'état de la grille, on distingue deux cas où l'on peut majoré beaucoup plus efficacement la complexité de l'algorithme \textbf{Voronoï}~\ref{algo:voronoï}:
\begin{itemize}
	\item $|S| \gg |A|$ : Dans ce cas, on peut majorer la complexité de l'algorithme \textbf{Voronoï}~\ref{algo:voronoï} en $\mathcal{O}{(J|S|log|S|)}$.
	\item $|S| \ll |A|$ : Dans ce cas, on peut majorer la complexité de l'algorithme \textbf{Voronoï}~\ref{algo:voronoï} en $\mathcal{O}{(J|A|)}$.
\end{itemize}
C'est deux(2) cas sont ceux qui sont le plus souvent rencontrés dans le jeu de \textbf{Tron}.\\

\subsection{Checker}
Compte tenu de la structure du jeu qui se joue sur une grille, nous pouvons considérer cette grille comme un échiquier, où les cases sont alternativement 
blanches et noires. En conséquence, un joueur ne peut se déplacer que d'une case blanche à une case noire ou vice versa. Ainsi, en utilisant l'algorithme 
\textbf{Voronoï}~\ref{algo:voronoï}, nous pouvons éliminer les cases blanches et noires excédentaires qui se trouvent dans la zone d'influence d'un joueur. 
Cela nous permet d'obtenir une limite supérieure plus précise de la zone d'influence d'un joueur.

\tocless\subsubsection{Implémentation}
L'algorithme \textbf{Checker}~\ref{algo:checker} est découle du développement réalisé sur celui de voronoï, une différence notable est que dans le case 
du \textbf{Checker}, nous comptons le nombre de cases blanches et noires dans la zone d'influence d'un joueur, ce qui permet d'en réduire le surplus.\\

\begin{algorithm}[H]
	\caption{Détermination de la zone d'influence d'un joueur}
	\label{algo:checker}
	\DontPrintSemicolon
	\SetKwComment{Comment}{/* }{ */}
	\KwIn{$state$ : Etat courant du jeu}
	\KwOut{$zones\_influence$ : Dictionnaire contenant la taille de la zone d'influence de chaque joueur}
	$zones\_influence \gets \{\}$\;
	$distances \gets \{\}$\;
	$count_{white} \gets 0$\;
	$count_{black} \gets 0$\;
	\For{$player \in state.players$}{
		$distances[player] \gets distance(state, player)$\;
		$zones\_influence[player] \gets 0$\;
		$count_{white}[player] \gets 0$\;
		$count_{black}[player] \gets 0$\;
	}
	\ForEach{$position\_vide \in state.vide\_positions$}{
		$min\_player \gets None$\;
		$min\_distance \gets \infty$\;
		\For{$player \in state.players$}{
			\If{$distances[player][position\_vide] < min\_distance$}{
				$min\_player \gets player$\;
				$min\_distance \gets distances[player][position\_vide]$ \Comment{On récupère le joueur le plus proche de la case vide}
			}
			\ElseIf{$distances[player][position\_vide] == min\_distance$}{
				$min\_player \gets None$\;
				$min\_distance \gets \infty$\;
			}
		}

		\If{$min\_player != None$}{
			$zones\_influence[min\_player] \gets zones\_influence[min\_player] + 1$\;
			\If{$position\_vide.color == 'white'$}{
				$count_{white}[min\_player] \gets count_{white}[min\_player] + 1$\;
			}
			\Else{
				$count_{black}[min\_player] \gets count_{black}[min\_player] + 1$\;
			}
			$zones\_influence[min\_player] \gets zones\_influence[min\_player] + 1$\;
		}
	}
	\For{$player \in state.players$}{
		$surplus \gets |count_{white}[player] - count_{black}[player]|$\;
		$zones\_influence[player] \gets zones\_influence[player] - surplus$\;
	}
	\Return $zones\_influence$\;
\end{algorithm}

\tocless\subsubsection{Complexité}
La complexité dans le pire des cas de l'algorithme \textbf{Checker}~\ref{algo:checker} est du même ordre que celle de l'algorithme \textbf{Voronoï}~\ref{algo:voronoï},
c'est à dire $\mathcal{O}{(J(|A| + |S|log|S|))}$. En effet, l'algorithme \textbf{Checker}~\ref{algo:checker} est une extension de l'algorithme \textbf{Voronoï}~\ref{algo:voronoï}. Cette 
extension permet de trouver une limite supérieure plus précise de la zone d'influence d'un joueur.\\


