\tocless\subsection{Problemématique de recherche}
Le jeu de Tron est un jeu multijoueur complexes qui implique la prise de décision stratégique,
la coordination et la compétition. L'un des principaux défis de ce jeu est d'anticiper les mouvements
des adversaires et de concevoir des stratégies efficaces qui peuvent maximiser les chances de survie et
de victoire d'un joueur. Cependant la stratégie optimale n'est pas toujours évidente, car elle dépend
de divers facteurs. En effet, la taille de la grille, le nombre d'adversaires, la profondeur de recherche, 
la stratégie des adversaires et les ressources de calcul sont autant de facteurs qui peuvent influencer
les performances d'un joueur. Ainsi, le projet vise à étudier l'efficacité de différents algorithmes de jeu
à travers l'analyse de l'impact de ces divers facteurs sur les performances des joueurs.\\

\tocless\subsection{Hypothèses de recherche}
Pour répondre à cette problématique, diverses hypothèses de recherche ont été formulées:
\begin{itemize}
	\item La performance des stratégies s'améliore avec la profondeur de recherche accordée à l'agent.
	\item L'efficacité des algorithmes se réduisent avec l'augmentation du nombre d'adversaires.
	\item Il existe une taille maximale de grille pour laquelle après dépassement, les performances des stratégies diminuent.
\end{itemize}
Ces hypothèses guideront les expérimentations, l'analyse des données et l'interpretation des résultats, et 
seront utilisées pour évaluer la pertinence des algorithmes de jeu.