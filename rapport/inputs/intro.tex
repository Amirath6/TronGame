La \textbf{théorie des jeux} et l'\textbf{intelligence artificielle} sont deux(2) domaines 
interconnectés qui ont connu une évolution fulgurante ces dernières années. La théorie des 
jeux est une branche des mathématiques qui se concentre sur l'analyse de la prise de décision 
stratégique dans des situations où de multiples acteurs interagissent et ont des intérêts 
contradictoires. Elle fournit un cadre pour l'analyse des choix des agents rationnels\footnote{Un 
agent rationnel est une entité qui vise toujours à réaliser des actions optimales sur la base de 
prémisses et d’informations données.} et la prédiction de leurs résultats dans différents scénarios. 
En revanche, l'intelligence artificielle vise à doter les machines de la capacité d'imiter le 
comportement humain. Elle repose sur diverses techniques, telles que les algorithmes de recherche, 
l'apprentissage automatique et le traitement du langage naturel, pour simuler un comportement 
intelligent et prendre des décisions sur la base de données et de règles.\\

Le mariage entre la théorie des jeux et l'intelligence artificielle a conduit à de nombreuses 
applications, telles que les moteurs d'échecs, les robots de poker et les systèmes de recommandation. 
De plus, cette alliance a permis l'émergence de nouvelles voies de recherche, notamment les systèmes 
multi-agents, l'apprentissage par renforcement et la théorie du choix social.\\ 

Dans ce projet, nous avons l'ambition d'utiliser la théorie des jeux et l'intelligence artificielle 
pour mettre en œuvre et étudier le \textbf{jeu de Tron} avec plusieurs joueurs et différentes stratégies. Le jeu de Tron 
est un jeu multijoueur captivant inspiré du célèbre jeu \textbf{Snake}\footnote{Le snake, de 
l’anglais signifiant « serpent », est un genre de jeu vidéo où le joueur contrôle un serpent 
qui s’agrandit au fur et à mesure de la progression du jeu, créant ainsi des obstacles pour le joueur}.
Dans ce dernier, les joueurs contrôlent un point qui se déplace sur une grille de taille fixe et laisse 
derrière lui un mur infranchissable à chaque déplacement. Le but est d'être le dernier joueur debout 
en évitant les collisions avec les murs, les bords et les trajectoires des adversaires. Ce jeu 
présente plusieurs défis pour l'analyse, tels que l'observabilité partielle, les mouvements 
simultanés et l'incertitude quant aux intentions des adversaires. Pour relever ces défis, les 
chercheurs ont proposé différents algorithmes, tels que $negamax$ avec élagage $alpha-beta$, 
recherche arborescente de $Monte$ $Carlo$ et $Max^n$. Toutefois, l'ajout de plusieurs joueurs et équipes 
crée de nouvelles dimensions de complexité et de stratégie.\\

L'objectif de ce projet est d'étudier l'efficacité de différents algorithmes de jeu en évaluant les 
performances d'abord d'un agent solitaire, puis d'un agent multi-joueur. Pour ce faire, 
une analyse statistique sera effectuée dans diverses configurations de jeu. Ainsi, 
nous pourrons déterminer les algorithmes les plus efficaces et les plus performants.\\

Dans ce document, nous introduirons d'abord le jeu de Tron. Puis, nous exposerons la structure du projet et 
les phases successives de son élaboration. Ultérieurement, nous approfondirons les algorithmes de jeu employés 
et l'architecture globale du projet. Finalement, nous présenterons les constatations dérivées de notre évaluation 
statistique et tirerons des conclusions sur la performance des algorithmes.